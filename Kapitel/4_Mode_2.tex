\chapter{Rendering Mode 2}
Dadurch, dass das Team nur aus zwei Personen besteht, wird nur auf die Rendering Modes 0 und 1 in dieser Studienarbeit näher eingegangen und vom Mode 2 die Grundlagen für das allgemeine Verständnis erklärt. \\
Im Mode 2 gibt des zwei Schichten mit dem Namen Background 2 und 3 welche beide skalierbar und rotierbar sind.
Das Prinzip im Mode 2 ist das Gleiche, wie im Mode 1 der BG2, nur dass im Mode 2 noch mehr Flexibilität durch zwei skalierbare Hintergründe gegeben ist. \\
\begin{table}[h]
\centering
\begin{tabular}{|l|l|l|l|l|}
\hline
\textbf{Mode}           & \textbf{BG 0}          & \textbf{BG1}           & \textbf{BG2}                & \textbf{BG3}                \\ \hline
\multicolumn{1}{|c|}{2} & \multicolumn{1}{c|}{-} & \multicolumn{1}{c|}{-} & \multicolumn{1}{c|}{affine} & \multicolumn{1}{c|}{affine} \\ \hline
\end{tabular}
\caption{Rendering Mode 2}
\label{Rendering Mode 2}
\end{table}



