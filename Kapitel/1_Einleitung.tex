\chapter{Rendering Modes}
Für die Rendering Modes 0, 1, 2 stehen vier übereinanderliegende Schichten. Diese Schichten haben die Bezeichnung Background 0 - Background 3.
Die Rendering Modes werden in einem speziellen Register festgelegt, dessen Namen \textbf{REG\_DISPCNT} trägt. In diesem Register werden die Modes und Background Features festgelegt, wie dann später die Ausgabe erfolgt.
Das Register hat die Adresse: \textbf{0x4000000}. Diese drei Modes arbeiten mit sogenannten Tiles (zu Deutsch: Kacheln). Diese Tiles werden in Hardware implementiert und brauchen weniger Speicher als die Bitmapgrafiken der Rendering Modes 4,5,6.\\
Die folgende Tabelle gibt einen kurzen Überblick der Modes 1,2,3 und beschreibt, wie sich die Background des jeweiligen Einstellung verhalten: \\
\begin{table}[h]
\centering
\begin{tabular}{|c|c|c|c|c|}
\hline
\textbf{mode} & \textbf{BG0} & \textbf{BG1} & \textbf{BG2} & \textbf{BG3} \\ \hline
1 & regular & regular & regular & regular \\ \hline
2 & regular & regular & affine & - \\ \hline
3 & - & - & affine & affine \\ \hline
\end{tabular}
\caption{Überblick der Hintergrund-Modi}
\label{Überblick der Hintergrund-Modi}
\end{table}
Es gibt zwei verschiedene Typen von Hintergründen. Regular Background werden oft als reine Texthintergründe bezeichnet, wohingegen die affinen Hintergründe die Fähigkeit der Rotation um die eigene Achse besitzen.
Es wird nun im folgenden anhand einiger Codebeispiele aus dem \textbf{Tonc\_code} näher auf die jeweiligen Modes der Tiles eingegangen und dessen Verhalten im mGBA beschrieben. \citep{mode}