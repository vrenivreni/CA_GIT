% ===========================================================
% Vorlage für einen Bericht 
% Erstelldatum: 21.08.2017
% ===========================================================

\documentclass[a4paper,11pt,bibliography=totoc,listof=totoc]{scrreprt}
%\documentclass{scrbook}
\usepackage[utf8]{inputenc}
\usepackage[T1]{fontenc}
\usepackage[ngerman]{babel}
\usepackage{tabularx}
\usepackage{listings}


% ===========================================================
% Korrekturrand von jeweils 3 cm links und rechts
% ===========================================================

\usepackage{geometry}
\geometry{a4paper,left=30mm,right=20mm,bottom=40mm}

% ===========================================================
% Optionale Pakete und Einstellungen
% ===========================================================

% \usepackage{scrhack}    % Sollte mit listings zusammen aktiviert werden
\usepackage{listings}   % Für Code-Beispiele
% \usepackage{multicol}   % Mehrspaltige Aufzählungen
\usepackage{amssymb}    % Mathematische Symbole
\usepackage{amsmath}    % Mathematisch Formeln
\usepackage{textcomp}  	  % Weitere Symbole
% \usepackage{ifsym}      % Weitere Symbole
\usepackage{url}          % URL Formatierung
\usepackage{wrapfig}


% \setcounter{tocdepth}{3}   % Tiefe der Überschriften die ins Inhaltsverzeichnis sollen
 \renewcommand{\baselinestretch}{1.25}\normalsize   % Zeilenabstand ändern, falls gewollt
\newcommand{\highlight}{\textbf}   % Zum hervorheben wichtiger Punkte \highlight{} (default: Fett) benutzen (Bei Änderungen nur hier und nicht im ganzen Dokument)

% ===========================================================
% Einfügen von PDF's und Bildern
% ===========================================================

\usepackage{ifpdf}
\usepackage[pdftex]{color}
\ifpdf
  \usepackage[pdftex]{graphicx}
\else
  \usepackage[dvips]{graphicx}
\fi

\usepackage{pdfpages}

% ===========================================================
% Literaturverzeichnis
% Muss in Alphabetischer Reihenfolge sein
% ===========================================================

% \usepackage[style=numeric,backend=biber, sorting=nyt,natbib=true]{biblatex} % Alternative Zitierung mit [x]
\usepackage[backend=bibtex, sorting=nyt,natbib=true]{biblatex} % Ähnlich Harvard-Citation style mit \autocite (übernommen aus "Hinweise zum Praktikumsbericht")
\usepackage[babel,german=guillemets]{csquotes}
\bibliography{Bib/Literaturverzeichnis}

% ===========================================================
% Änderung der Überschrift des Literaturverzeichnisses
% ===========================================================

\DefineBibliographyStrings{ngerman}{
    bibliography = {Literaturverzeichnis}
}

% ===========================================================
% Für Abkürzungen und Abkürzungsverzeichnis
% ===========================================================

\usepackage[printonlyused]{acronym}

% ===========================================================
% Inhaltsverzeichns
% ===========================================================

\usepackage[colorlinks=true,linkcolor=black, citecolor=black]{hyperref}
\usepackage[all]{hypcap}

% ===========================================================
% Dokument Anfang
% ===========================================================

\begin{document}

% ===========================================================
% Abstand zwischen Literatureinträgen (muss hier stehen)
% ===========================================================

\setlength{\bibitemsep}{12pt} 

% ===========================================================
% NUR ZU TESTZWECKEN!! UNGENUTZTE LITERATUR DARF NICHT VORKOMMEN AM ENDE! Danach kommentieren!!!!!!!!!!!!!! <<<<<<<<<<<<<<<<<<<<<<<<<<<<
  \nocite{*} % Ganze Literaturliste anzeigen, auch ungenutzte
% ===========================================================

% ===========================================================
% Deckblatt
% ===========================================================
\begin{titlepage}

\includegraphics[height=10mm]{img/OTHLogo.jpg}

\vspace{10mm}

\begin{center}
\Huge Studienarbeit \\ 
\large \ \\
\Large Rendering-Modes 0,1,2 eines Gameboy Advanced Emulators 

\vspace{20mm}

%\includegraphics{img/mgba.png} \\
im Fach \\
\textbf{Computerarchitektur} \\

Datum: 07.07.\the\year
\vspace{70mm}

\end{center}
\begin{table}[h]
\centering
\begin{tabular}{|l|l|l|}
\hline
\textbf{Name}    & \textbf{Matrikelnummer} & \textbf{Thema in der Arbeit} \\ \hline
Verena Schmöller & 3029620                 & Mode 0                       \\ \hline
Peter Reu        & 3032132                 & Mode 1, 2                    \\ \hline
\end{tabular}
\caption{Teammitglieder}
\label{Teammitglieder}
\end{table}






\end{titlepage}

% ===========================================================
% Inhaltsverzeichnis
% ===========================================================

\renewcommand*\chapterheadstartvskip{\vspace*{-\topskip}}

\pagenumbering{Roman} 
\tableofcontents

% \renewcommand*\chapterheadstartvskip{\vspace*{15mm}}

% ===========================================================
% Abbildungsverzeichnis
% ===========================================================

\listoffigures

% ===========================================================
% Abkürzungsverzeichnis
% ===========================================================
\listoftables
% ===========================================================
% Tabellenverzeichnis
% ===========================================================

%% Muss von Hand sortiert werden!!

\addchap{Abkürzungsverzeichnis} %Abkürzungsverzeichnis
\markboth{Abkürzungsverzeichnis}{}
\begin{acronym}[GBA] 			% längste Abkürzung steht in eckigen Klammern
  \acro{GBA}{Gameboy Advanced Emulator}
  \acro{IDA}{Interaktiver Disassemlber}
\end{acronym}


% ===========================================================
% Bericht Teile - Anpassen je nach Notwendigkeit
% ===========================================================

\clearpage
\pagenumbering{arabic}    

\chapter{Rendering Modes}
Für die Rendering Modes 0, 1, 2 stehen vier übereinanderliegende Schichten. Diese Schichten haben die Bezeichnung Background 0 - Background 3.
Die Rendering Modes werden in einem speziellen Register festgelegt, dessen Namen \textbf{REG\_DISPCNT} trägt. In diesem Register werden die Modes und Background Features festgelegt, wie dann später die Ausgabe erfolgt.
Das Register hat die Adresse: \textbf{0x4000000}. Diese drei Modes arbeiten mit sogenannten Tiles (zu Deutsch: Kacheln). Diese Tiles werden in Hardware implementiert und brauchen weniger Speicher als die Bitmapgrafiken der Rendering Modes 4,5,6.\\
Die folgende Tabelle gibt einen kurzen Überblick der Modes 1,2,3 und beschreibt, wie sich die Background des jeweiligen Einstellung verhalten: \\
\begin{table}[h]
\centering
\begin{tabular}{|c|c|c|c|c|}
\hline
\textbf{mode} & \textbf{BG0} & \textbf{BG1} & \textbf{BG2} & \textbf{BG3} \\ \hline
1 & regular & regular & regular & regular \\ \hline
2 & regular & regular & affine & - \\ \hline
3 & - & - & affine & affine \\ \hline
\end{tabular}
\caption{Überblick der Hintergrund-Modi}
\label{Überblick der Hintergrund-Modi}
\end{table}
Es gibt zwei verschiedene Typen von Hintergründen. Regular Background werden oft als reine Texthintergründe bezeichnet, wohingegen die affinen Hintergründe die Fähigkeit der Rotation um die eigene Achse besitzen.
Es wird nun im folgenden anhand einiger Codebeispiele aus dem \textbf{Tonc\_code} näher auf die jeweiligen Modes der Tiles eingegangen und dessen Verhalten im mGBA beschrieben. \citep{mode}
\chapter{Rendering Mode 0}
Wie oben bereits erwähnt, handelt es sich bei dem Modus 0 um eine Einstellung des \ac{GBA}, welche 4 reguläre Hintergründe anzeigen kann. Reguläre Hintergründe sind weder skalierbar, noch rotierbar. Das ist der Unterschied zu den affinen Hintergründen. Angezeigt werden die verschiedenen Karten anhand der eingestellten Priorität, wobei die niedrigste Nummer der höchsten Priorität entspricht. Die unterschiedlichen Hintergünde können entweder sogenannte Tilemaps sein, welche sich aus Kacheln zusammensetzen, oder Sprites, also Objekte oder Bilder. Zuerst wird auf die Tilemaps und Kacheln genauer eingegangen. \citep{mode}
\section{Tilemaps}
\begin{wrapfigure}{R}{0mm}
	\includegraphics[height=40mm]{img/tile_blue.png}
	\caption{Beispieltile}
	\label{example_tile}
\end{wrapfigure}
Jede Kachel besteht standardmäßig aus 16x16 Pixel. Diese Zusammensetzung aus Pixeln wird in einem Array aus Hexadezimalen, dem Tileset, gespeichert. So wird in einem externen Dokument in zwei Schritten der Hintergrund aufgebaut. Zuerst werden die Pixel der Reihe nach in dem Tileset gespeichert. Danach wird der Hintergrund aus den vorher angelegten Blöcken zusammengesetzt. Es entsteht eine Matrix, die sogenannte Tilemap. 

Die Blöcke werden allerdings nicht als 16x16 Pixel große Bitfolgen abgespeichert, sondern nochmal aufgeteilt in vier 8x8 Pixel große Objekte. Das hat den Vorteil, dass diese Bauteile meist wiederverwendbar sind, und so nochmal Speicherplatz gespart wird. Das kann zum Beispiel der Fall sein, wenn eine Kachel nur aus einem Farbton besteht. Dann muss nur ein einziger 8x8 Baustein abgespeichert werden und kann vier mal eingesetzt werden. Die Zusammensetzung eines Blockes wird im Folgenden an anhand des Beispiels in Abbildung \ref{tile_quarter} dargestellt.
\begin{figure}[H]
	\centering
	\begin{subfigure}{.5\textwidth}
		\centering
		\includegraphics[height=40mm]{img/tile_pixels.png}
		\caption{Pixelabfolge der einzelnen Bestandteile}
	\end{subfigure}%
	\begin{subfigure}{.5\textwidth}
		\centering
		\includegraphics[height=40mm]{img/tile_blue_quarter.png}
		\caption{Aufteilung eines Tiles in vier 8x8 Pixel Blöcke}
	\end{subfigure}
	\caption[Bestandteile eines Blocks]{}
	\label{tile_quarter}
\end{figure}
Jedes Pixel wird als eine Stelle in der Hexadezimalen, also in 4 Bit, abgespeichert. Die Reihenfolge der Pixel erfolgt Zeilenweise, genauso wie die Abfolge der Kacheln selbst. So kann ein Block in der Tilemap nicht einfach der Reihe nach referenziert werden, sondern die obere Hälfte muss in der ersten Zeile abgerufen werden, die untere in der darauf folgenden. Außerdem enthält ein Screenblock bei dem \ac{GBA} noch zusätzliche Informationen, welche in Tabelle \ref{tile_memory} dargestellt sind.
\begin{table}[H]
\centering
\begin{tabular}{|p{1cm}|p{1cm}|p{4cm}|p{9cm}|}
\hline
\textbf{Bits} & \textbf{Name} & \textbf{Definition} & \textbf{Beschreibung} \\ \hline
0 - 9 & TID & SE\_ID\# & Blockindex des Screen-entries (SE) \\ \hline
A - B & HF, VF & SE\_HFLIP, SE\_VFLIP. SE\_FLIP\# & Horizontale bzw. Vertikale Spiegelung \\ \hline
C - F & PB & SE\_PALBANK\# & Palette welche im 16-Farben Modus benutzt wird. Hat keinen Effekt für 256-Farben Hintergründe. \\ \hline
\end{tabular}
\caption{Einstellungen eines Blockes \citep{tile}}
\label{tile_memory}
\end{table}

\begin{wrapfigure}{L}{0mm}
	\includegraphics[height=30mm]{img/tiles_palette.png}
	\caption{Anwendung verschiedener Paletten}
	\label{tile_palettes}
	\includegraphics[height=30mm]{img/tiles_palette_quarter.png}
	\caption{Hex-Codes der Paletten}
	\label{tile_palettes_hex}
\end{wrapfigure}
Um diese Zusammensetzung beispielhaft darzustellen, kann man auf den Blöcken links in den Abb. \ref{tile_palettes} und \ref{tile_palettes_hex} die Verwendung verschiedener Paletten sehen. 

Die Palette wird ebenfalls extern angelegt und im Screenblock, entsprechend der oberen Tabelle, referenziert. Eine Palette wird nur bei einer Bittiefe von 4 (16 Farben, 16 Unterpaletten) angelegt, nicht aber bei 8 (256 Farben / 1 Palette), die entsprechenden Bits werden dann auf 0 gesetzt. 

Im \textbf{\ac{VRAM}} ist Platz für 32 Screenblocks um die Tilemaps zu speichern. Durch die Länge eines jeden Screenblocks ergibt sich, dass insgesamt 32x32 \ac{SE} hineinpassen. Das heißt, es kann maximal eine 256x256 Pixel große Karte gespeichert werden. \citep{gbatek} Die größeren Karten verwenden einfach mehr als einen Screenblock. Der in \textbf{REG\_BGxCNT} gesetzte \ac{SE}-Index ist der Bildschirmbasisblock, der den Beginn der Tilemap angibt. Größere Hintergründe verwenden nicht einfach mehrere Bildschirmblöcke, sondern sie werden als vier separate Karten aufgerufen. Jeder nummerierte Block ist ein Kontingentblock im Speicher. Dies bedeutet, um den \ac{SE}-Index zu erhalten, braucht man zuerst den Screenblock, in dem man sich befindet und dann die \ac{SE}-Nummer in diesem Screenblock. \citep{tile}

Die Zusammensetzung der einzelnen Screenblock-Layouts erfolgt wiederum der Reihe nach. Im Programm selbst wiederholen sich die jeweiligen Screenblocks immer wieder, sodass ein unendlich großer Hintergrund simuliert wird und die "Spiel-Welt" sozusagen nie zu Ende ist. Die unten aufgeführte Karte in Abb. \ref{tilemap} ist aus zwei Screenblocks in einem 64x32 Layout dargestellt. Der Code darunter veranschaulicht das Laden der einzelnen Komponenten in das Spiel. Hier werden zunächste die einzelnen Komponenten, wie die Palette, die Kacheln und die Tilemap in den \textbf{\ac{VRAM}} geladen und in \textbf{REG\_BG0CNT} zu einem Hintergrundobjekt zusammengeführt. Danach wird der Rendering Mode, in diesem Fall mit \textbf{DCNT\_MODE0} auf 0 gesetzt. Die while-Schleife im Anschluss ermöglicht das unbegrenzte scrollen durch den Bildschirm mit den Pfeiltasten.
\begin{figure}[h]
	\centering
	\includegraphics[height=70mm]{img/tilemap.png}
	\caption{Beispiel einer 64 x 32 Tilemap}
	\label{tilemap}
\end{figure}

\lstinputlisting[language=C, frame=single, breaklines = true, numbers=left, basicstyle=\ttfamily, keywordstyle=\color{blue}\ttfamily, stringstyle=\color{red}\ttfamily, commentstyle=\color{green}\ttfamily, label=tilemap_code,caption=main() Funktion]{code/tilemap.c}
\section{Sprites}
\begin{wrapfigure}{R}{0mm}
	\includegraphics[height=40mm]{img/sprites.png}
	\caption{Beispiel Sprite und Aufteilung in einzelne Kacheln}
	\label{sprites}
\end{wrapfigure}
Unter Sprites versteht man grundsätzlich Hintergrundbilder. Meist werden sie als Dekoration in den Hintergrund eingebettet, da sie nicht das komplette Display einnehmen und so an beliebigen Stellen eingefügt werden können. Die Sprites sind wiederum als eine Zusammensetzung aus 8x8 Pixel großen Bausteinen gespeichert, wie Kacheln. Das ist in Abbildung \ref{sprites} beispielhaft dargestellt.

Der Hauptunterschied zwischen Kacheln (Tiles) und Sprites liegt darin, dass die Tilemap den gesamten Hintergrund bedeckt. Das heißt, an jeder Stelle des Hintergrundes wird ein Pixel bzw. Ein Block abgespeichert. Dagegen ist ein Sprite eine eigene Zusammensetzung aus Blöcken, die an einer festen Stelle im Hintergrund angezeigt wird. 

Die Spezifikationen eines Sprites werden in dem sogenannten \textbf{\ac{OAM}} gespeichert. Dieser besteht aus zwei Strukturen: Der \textbf{OBJ\_ATTR}-Struktur, welche alle regulären Attribute enthält, und der \textbf{OBJ\_AFFINE}-Struktur, welche potentielle Transformationen enthält.
Die \textbf{OBJ\_ATTR}-Struktur besteht aus drei 16-Bit-Attributen für Eigenschaften wie Größe, Form, Grundblock etc. Diese drei Attribute sind im Folgenden genauer erläutert.
\paragraph{Attribut 0}
%\subsection{Attribut 0}
Das erste Attribut enthält einige Informationen, aber die wichtigsten Bestandteile hier sind die y-Koordinate und die Form des Sprites. Wichtig ist auch, ob das Sprite transformierbar ist (ein affines Sprite) und ob die Kacheln eine Bittiefe von 4 (16 Farben, 16 Unterpaletten) oder 8 (256 Farben / 1 Palette) haben.
%\begin{table}[ht]
%\centering
%\begin{tabular}{|p{1cm}|p{1cm}|p{3.5cm}|p{9cm}|}
%\hline
%\textbf{Bits} & \textbf{Name} & \textbf{Definition} & \textbf{Beschreibung} \\ \hline
%0 - 7 & Y & ATTR0\_Y\# & Y-Koordinate (Position des Sprites) \\ \hline
%8 - 9 & OM & ATTR0\_REG, ATTR0\_AFF, ATTR0\_HIDE, ATTR0\_AFF\_DBL. ATTR0\_MODE\# & Modus des Sprites: 
%{\begin{itemize}
%	\setlength\itemsep{0em}
%	\item 00: Normale Anzeige
%	\item 01: Affine Anzeige - Matrix wird in Attr. 1 angegeben
%	\item 10: Keine Anzeige (versteckt)
%	\item 11: Affine Doppelanzeige
%\end{itemize}} \\ \hline
%A - B & GM & ATTR0\_BLEND, ATTR0\_WIN. ATTR0\_GFX\# & Spezialeffekte:
%{\begin{itemize}
%	\setlength\itemsep{0em}
%	\item00: Normale Anzeige
%	\item01: Alpha Blending
%	\item10: Keine Anzeige, aber ist Teil des Bildschirms
%	\item11: Verboten
%\end{itemize}} \\ \hline
%C & Mos & ATTR0\_MOSAIC & Mosaik Effekt \\ \hline
%D & CM & ATTR0\_4BPP, ATTR0\_8BPP & Farbauswahl \\ \hline
%E - F & Sh & ATTR0\_SQUARE, ATTR0\_WIDE, ATTR0\_TALL. ATTR0\_SHAPE\# & Form des Sprites \\ \hline
%\end{tabular}
%\caption{Zusammensetzung von Attribut 0 eines Sprites}
%\label{sprite-attr0}
%\end{table}
\paragraph{Attribut 1}
%\subsection{Attribut 1}
Die wichtigsten Bestandteile dieses Attributs sind die X-Koordinate und die Größe des Sprites. Die Rolle der Bits 8 bis 14 hängt davon ab, ob dies ein affiner Sprite ist. Wenn dies der Fall ist, geben diese Bits an, welcher der 32 OBJ\_AFFINE verwendet werden soll. Wenn nicht, enthalten sie sogenannte flipping flags, welche eine horizontale bzw. vertikale Spiegelung des Sprites zur Folge haben.
%\begin{table}[ht]
%\centering
%\begin{tabular}{|p{1cm}|p{1cm}|p{3.5cm}|p{9cm}|}
%\hline
%\textbf{Bits} & \textbf{Name} & \textbf{Definition} & \textbf{Beschreibung} \\ \hline
%0 - 8 & X & ATTR1\_Y\# & X-Koordinate (Position des Sprites) \\ \hline
%9 - D & AID & ATTR1\_AFF & Affiner Index, nur gültig falls Affine Flag in Attribut 0 gesetzt ist \\ \hline
%C - D & HF, VF & ATTR1\_HFLIP, ATTR1\_VFLIP. ATTR1\_FLIP\# & Horizontale bzw. Vertikale Spiegelung, nur gültig falls Affine Flag in Attribut 0 nicht gesetzt ist \\ \hline
%E - F & Sz & ATTR1\_SIZE & Größe des Sprites \\ \hline
%\end{tabular}
%\caption{Zusammensetzung von Attribut 1 eines Sprites}
%\label{sprite-attr1}
%\end{table}
\paragraph{Attribut 2} 
%\subsection{Attribut 2}
Dieses Attribut enthält Informationen darüber, welche Kacheln angezeigt und welche Priorität diese bezogen auf die Hintergründe haben. Falls dies ein 4bpp-Sprite ist, wird hier auch die Farbpalette angegeben. \citep{sprites}
%\begin{table}[ht]
%\centering
%\begin{tabular}{|p{1cm}|p{1cm}|p{3.5cm}|p{9cm}|}
%\hline
%\textbf{Bits} & \textbf{Name} & \textbf{Definition} & \textbf{Beschreibung} \\ \hline
%0 - 9 & TID & ATTR2\_ID\# & Index des Basisblocks \\ \hline
%A - B & Pr & ATTR1\_PRIO\# & Anzeigepriorität \\ \hline
%C - F & PB & ATTR1\_PALBANK\# & Palette für 16-Farbmodus \\ \hline
%\end{tabular}
%\caption{Zusammensetzung von Attribut 2 eines Sprites}
%\label{sprite-attr2}
%\end{table}

Die Initialisierung eines Sprites wird anhand eines kurzen Code-Ausschnitts in Listing \ref{init_sprite} erläutert. Hier werden zuerst der Index der ersten Kachel und die Palette zwischengespeichert. Dann werden der Reihe nach alle drei Attribute so gesetzt, dass nur die wichtigsten Bestandteile verändert werden müssen, die restlichen Bits bleiben demnach 0. In der main()-Funktion wird anschließend wieder der Hintergrund initialisiert, ähnlich wie in Listing \ref{tilemap_code}.
\lstinputlisting[language=C, frame=single, breaklines = true, numbers=left, basicstyle=\ttfamily, keywordstyle=\color{blue}\ttfamily, stringstyle=\color{red}\ttfamily, commentstyle=\color{green}\ttfamily, label=init_sprite,caption=Sprite Initialisierung]{code/init_sprite.c}



\chapter{Rendering Mode 1}
Erklärung des Mode 1 Renderings am Beispiel \textbf{sbb\_aff.gba} vom \textbf{Tonc\_code}. Es wird lediglich auf einen Teil des Beispielcodes eingegangen, da ein vollständige Erklärung den Rahmen dieser Arbeit sprengen würden.
Im Mode 1 hat der GBA nicht nur reguläre Backgrounds, sondern auch einen affinen Background. 
Dieser affine Background ermöglicht es, eine geometrische Transformation um einen Punkt zu erreichen. Dies sind unter anderem eine Skalierung und/oder ein Rotation des Hintergrundes. Diese speziellen Hintergründe können, je nach Einstellung im Code verschiedene Größen vorweisen, welche über \textbf{DEFINES} vorgegeben sind. Folgende Tabelle soll dies veranschaulichen:
\begin{table}[ht]
\centering
\begin{tabular}{|c|c|c|c|}
\hline
\textbf{Size} & \textbf{Define} & \textbf{Tiles} & \textbf{Pixels} \\ \hline
00 & BG\_AFF\_16x16 & 16x16 & 128x128 \\ \hline
01 & BG\_AFF\_32x32 & 32x32 & 256x256 \\ \hline
10 & BG\_AFF\_64x64 & 64x64 & 512x512 \\ \hline
11 & BG\_AFF\_128x128 & 128x128 & 1024x1024 \\ \hline
\end{tabular}
\caption{Einstellungen des affinen Hintergrundes}
\label{size}
\end{table}
\begin{wrapfigure}{l}{0mm}
	\includegraphics[height=50mm]{img/pic128.png}
	\caption{Bilddarstellung mit 128x128 Pixel}
	\includegraphics[height=50mm]{img/pic16.png}
	\caption{Bilddarstellung mit 128x128 Pixel}
\end{wrapfigure}
Um dies beispielhaft zu demonstrieren, wird die vorgegebenen Einstellung von \textbf{BG\_AFF\_64x64} auf \textbf{BG\_AFF\_128x128} und \textbf{BG\_AFF\_16x16 }geändert und man kann eine deutliche Veränderung hinsichtlich der Anzahl der Kacheln beobachten. Diese haben sich in der Anzahl der in X-Achse befindlichen Tiles auf 128 erhöht, beziehungsweise auf 16 verringert. Das obere Bild zeigt das die Einstellung von 128x128 Tiles, das untere die Einstellung von 16x16 Tiles. Je nach Setzen des Wertes der global definierten Variable für die Größe des skalierbaren Hintergrund ändert sich auf die Anzahl der farbigen Kacheln. Im Beispielbild mit 16x16 als Konfiguration, sind nur die roten Kacheln zu sehen, wohingegen bei 128x128 mehrere Farben zu beobachten sind, jedoch nur immer zwei Spalten mit der jeweiligen gleichen Farbe. Damit man mehrere Spalten mit der gleichen Farbe sehen kann, muss die globale Variable mit dem Namen \textbf{MAP\_AFF\_SIZE} geändert werden. Diese hat als Standardwert 0x0100. Wird dieser auf den Wert 0x0200 gesetzt, erhöht sich die Anzahl der gleichfarbigen Kachel von zwei auf vier im Beispiel mit der Konfiguration von 128x128 Tiles. Im Codebeispiel \textbf{sbb\_aff.c} wird der Hintergrund mit dem Funktion \textbf{init\_map()} eingestellt und dargestellt. Auf diese Funktion wird später nochmals näher eingegangen und die Codeabschnitte veranschaulicht erklär

\section{Affine Background Control Register}
Das primäre Register, welchen den Hintergrund identifiziert, hat die Definition REG\_BGxCNT. Das \textbf{X} ist ein Platzhalter für den Hintergrund, der angesprochen wird. Im Mode 1 wäre das \textbf{REG\_BG2CNT} der skalierbare Hintergrund 2. Dieses Register hat die Länge zwei und die Adresse \textbf{ 0400:0008h + 2*x} . Das Register hat folgende Spezifikationen:
\begin{table}[ht]
\centering
\begin{tabular}{|c|c|c|c|c|c|c|c|}
\hline
\textbf{F E} & \textbf{D} & \textbf{C B A 9 8} & \textbf{7} & \textbf{6} & \textbf{5 4} & \textbf{3 2} & \textbf{1 0} \\ \hline
Sz & Wr & SBB & CM & Mos & - & CBB & Pr \\ \hline
\end{tabular}
\caption{REG\_BGxCNT Spezifikationen}
\label{regspec}
\end{table}

\newpage

\begin{table}[h!]
\centering
\begin{tabular}{|p{1cm}|p{1cm}|p{4cm}|p{9cm}|}
\hline
\textbf{Bits} & \textbf{Name} & \textbf{Definition} & \textbf{Beschreibung} \\ \hline
0 - 1 & Pr & BG\_PRIO\# & \textbf{Priority}: Legt die Reihenfolge für das Zeichnen der Hintergründe fest \\ \hline
2 - 3 & CBB & BG\_CBB\# & \textbf{Character Base Block}: Setzt den Charblock der als Basis für den Index der Tiles dient. Mögliche Werte 0 - 3 \\ \hline
6 & Mos & BG\_MOSAIC & \textbf{Mosaic}: Aktiviert den Mosaic Effekt \\ \hline
7 & CM & \begin{tabular}[c]{@{}l@{}}BG\_4BPP,\\ BG\_8PP\end{tabular} & \textbf{Color Mode}: zu Deutsch Farbmodus. Wenn das Bit gesetzt ist, dann 256 Farben, wenn nicht gesetzt, dann 16 verschiedene Farben \\ \hline
8 - C & SBB & BG\_SBB\# & \textbf{Screen Base Block}: Setzt den Bildschirmblock der als Basis für den Index der Map/Karte dient. Mögliche Werte 0 - 31 \\ \hline
D & Wr & BG\_WRAP & \textbf{Affine Wrapping}: Falls gesetzt, dann rotieren die affinen Hintergründe um ihre Kanten. Diese Einstellung hat jedoch keine Auswirkung auf reguläre Hintergründe \\ \hline
E - F & Sz & BG\_SIZE & \textbf{Background Size}: hier wird die Hintergrundgröße eingestellt. \\ \hline
\end{tabular}
\caption{REG\_BGxCNT Detailbeschreibung}
\label{regspecdetail}
\end{table}

Um die Tabelle näher zu erläutern, wird in einem Beispiel der Wert \textbf{BG\_WRAP} gesetzt und danach beschrieben, welche Auswirkung dies hat. Der Background 2, welche für die Darstellung der skalierbaren Kacheln zuständig ist, hat den kompletten Bildschirm des mGBA eingenommen. Lässt man den Wert weg, sieht man wieder einen schwarzen Hintergrund. Mit \textbf{BG\_WRAP} ist dies nicht mehr möglich, weil der Hintergrund dann die Kacheln sind. Die Kachel der eigentlich rotierbaren Ebene füllen den kompletten Bildschrim aus. Man kann dies mit einem Desktophintergrund vergleichen, welcher den kompletten Bildschirm belegt. 
Bemerkung: BG\_WRAP ist nur bei rotier- und skalierbaren Hintergründen möglich. Bei regulären Hintergründen funktioniert diese Einstellung nicht!

\newpage
\section{init\_map() Funktion}
In diesem Abschnitt wird die \textbf{init\_map()} Funktion nähert behandelt, welche für das Initialisieren und dem Zeichnen der Kacheln zuständig ist. \\
Zuerst das Code Snippet:

\lstinputlisting[language=C, frame=single, breaklines = true, numbers=left, basicstyle=\ttfamily, keywordstyle=\color{blue}\ttfamily, stringstyle=\color{red}\ttfamily, commentstyle=\color{green}\ttfamily, label=initmap,caption=init\_map() Funktion]{code/initmap.c}

Die Variable \textbf{ses} ist der Platzhalter für die Beschreibung der Farben der jeweiligen Tiles, die dann angezeigt werden. In der folgenden Tabelle wird das Farbspektrum und die Ausgabe der Farben näher beschrieben. Dies ist für das Verständnis wichtig, wie der mGBA die Farben berechnet und ausgibt. In diesem Standardbeispiel sind es acht verschieden Farben die jeweils immer über vier Spalten und Zeilen gemeinsam ausgegeben werden. Da die Iteration über 16 Werte geht, wird ab Wert 0x08 wieder die gleiche Farbe, wie bei 0x01 angezeigt.

\begin{table}[h]
\centering
\resizebox{\textwidth}{!}{%
\begin{tabular}{|l|l|l|l|l|l|l|l|l|}
\hline
\textbf{Hex}   & \begin{tabular}[c]{@{}l@{}}0x1, \\ 0x9\end{tabular}               & \begin{tabular}[c]{@{}l@{}}0x2, \\ 0xA\end{tabular}                & \begin{tabular}[c]{@{}l@{}}0x3, \\ 0xB\end{tabular}                & \begin{tabular}[c]{@{}l@{}}0x4, \\ 0xC\end{tabular}                & \begin{tabular}[c]{@{}l@{}}0x5, \\ 0xD\end{tabular}               & \begin{tabular}[c]{@{}l@{}}0x6, \\ 0xE\end{tabular}                  & \begin{tabular}[c]{@{}l@{}}0x7, \\ 0xF\end{tabular}                 & \begin{tabular}[c]{@{}l@{}}0x8, \\ 0x10\end{tabular}                 \\ \hline
\textbf{Farbe} & \begin{tabular}[c]{@{}l@{}}Rot mit \\ grauen Schrift\end{tabular} & \begin{tabular}[c]{@{}l@{}}Grün mit \\ pinker Schrift\end{tabular} & \begin{tabular}[c]{@{}l@{}}Geld mit \\ blauer Schrift\end{tabular} & \begin{tabular}[c]{@{}l@{}}Blau mit \\ gelber Schrift\end{tabular} & \begin{tabular}[c]{@{}l@{}}Pink mit \\ grüner Schrit\end{tabular} & \begin{tabular}[c]{@{}l@{}}Bläulich mit\\ roter Schrift\end{tabular} & \begin{tabular}[c]{@{}l@{}}Weiß mit \\ oranger Schrift\end{tabular} & \begin{tabular}[c]{@{}l@{}}Orange mit \\ weißer Schrift\end{tabular} \\ \hline
\end{tabular}%
}
\caption{Farbdarstellung der Tiles}
\label{tilesfarbe}
\end{table}
\newpage
\begin{wrapfigure}{r}{0mm}
	\includegraphics[height=50mm]{img/gerade.png}
	\caption{Anzeige der geraden Kacheln}
	\includegraphics[height=50mm]{img/ungerade.png}
	\caption{Anzeige der ungeraden Kacheln}
	\includegraphics[height=50mm]{img/mapaffsize.png}
	\caption{MAP\_AFF\_SIZE/4 Darstellung}
\end{wrapfigure}
Der achtstellige Hexadezimalwert von der Variable ses lässt sich in vier Kategorien einteilen, die dann die jeweiligen Spalten darstellen
Als Beispiel hierfür, werden einmal die geraden Spalten und die ungeraden Spalten mit unterschiedlichen Farben angezeigt. 
An diesen beiden Bilder erkennt man, dass einmal die geraden Spalten mit rot und und pink belegt sind und die ungeraden mit gelb und bläulicher Farbe. Dies kann man den Hexwerten von \textbf{ses = 0x000D0001} entnehmen. Die letzte eins entspricht der ersten Spalte bei der Anzeige. Beim unteren Bild erkennt man schön den Spaltenversatz zum obigen Bild. Das resultiert daraus, dass hier nur die ungeraden Spalten angesprochen werden. Der Hexadezimalwert ist im unteren Beispiel mit dem Wert \textbf{ses = 0x0E000300} versehen. Diese Beispiel zeigt, dass man in der Lage ist, die einzelne Spalten beliebig farblich zu verändern und anzupassen. Darüber hinaus sieht man im Code - Snippet \textbf{ses += 0x0000000}. Hier können die Farben auch angepasst werden, nur dass die sich dann Zeilenweise verändern. Zum Beispiel ist es dabei möglich zwei Zeilen mit rot zu belegen, danach dann zwei Zeilen mit einer anderen Farbe. Man hat auch die möglichkeit vier Zeilen mit einer anderen Farbe zu belegen oder jede Zeile.Dies ist abhängig anhander der Einstellung die man wählt. Hier spielt dann die Variable \textbf{pse += MAP\_AFF\_SIZE/4} eine nützliche Rolle. Bei \textbf{/4} werden vier Zeilen jeweils mit der gleichen Farbe angezeigt. Bei \textbf{/8} werden lediglich nur zwei Zeilen mit der gleichen Farbe versehen.


\section{Skalierbare und rotierbarer Hintergrund}
Die Funktion sbb\_aff() übernimmt die komplette Aufgabe bezüglich Rotierbarkeit und Skalierung der Tiles, welche von der Funktion init\_map() gezeichnet werden. Im Folgenden wird auf die Funktion sbb\_aff() näher eingegangen und diese ausführlich erläutert. Die Funktion übernimmt unter anderem das Zeichen der $p_{0}$, $q_{0}$ und dx Variablen beziehungsweise Koordinanten, welche im mGBA unten rechts angezeigt werden. Hier wird lediglich eine print-Funktion aufgerufen, welche die aktuellen Werte ausliest und eine Ausgabe ermöglicht. Die print-Funktion wird hier nicht näher erläutert, sondern die Konzentration liegt bei der Skalierung und Rotierbarkeit der Karte. Die affinen Tilemaps benutzen andere Register, welche für diese Dynamik ausgelegt sind. Anstatt \textbf{REG\_BGxHOFS} und \textbf{REG\_BGxVOFS} benutzen die affinen Hintergründe die Register \textbf{REG\_BGx}X und \textbf{REG\_BGxY,} wobei hier \textbf{x} für den Hintergrund steht. Im folgenden wird näher auf die Register und auf die Transformation der affinen Hintergründe eingegangen. Es werden auch anschaulich die mathematischen Operationen aufgezeigt. Die Register sind in zwei unterschiedliche Typen aufgeteilt. Das eine ist für den Vektor dx (\textbf{REG\_BGxX} und \textbf{REG\_BGxY}) zuständig, das andere für die affine \textit{Matrix P} (\textbf{REG\_BGxPA} und \textbf{REG\_BGxPG}) zuständig. Auf die Matrix P wird aufgrund des Umfanges nicht näher darauf eingegangen.
\begin{table}[h]
\centering
\begin{tabular}{|c|c|c|}
\hline
\textbf{Register}                                                & \textbf{Länge} & \textbf{Adresse}      \\ \hline
REG\_BGxCNT                                                      & 2              & 0400:0008h + 2*x      \\ \hline
\begin{tabular}[c]{@{}c@{}}REG\_BGxPA,\\ REG\_BGxPD\end{tabular} & 2              & 0400:0020h + 10*(x-2) \\ \hline
REG\_BGxX                                                        & 4              & 0400:0028h + 10*(x-2) \\ \hline
REG\_BGxY                                                        & 4              & 0400:002Ch + 10*(x-2) \\ \hline
\end{tabular}
\caption{Register der affinen Hintergruende}
\label{affineregister}
\end{table}


\subsection{Transformation der Tilemap}
Die Variable dx beinhaltet die Hintergrundkoordinaten zum Zeitpunkt der Initialisierung der Map.
Die Matrix P beschreibt die Transformation (Rotierung) der X und Y Koordinaten der Map. Dies ist wie folgt mathematisch definiert:
\begin{equation}
T(dx)p := p + dx
\end{equation}
\begin{equation}
T^{-1}dx = T(-dx)
\end{equation}
\begin{equation}
P = A^{-1}
\end{equation}
\begin{equation}
T(dx)q = p
\end{equation}
\begin{equation}
P * q = p
\end{equation}
\begin{equation}
q = A * T(-dx)p
\end{equation}
\begin{equation}
T(dx)P * q = p
\end{equation}
\begin{equation}
P * (q-q_{0}) = (p-p_{0})
\end{equation}
\begin{equation}
dx + P * q_{0} = p_{0}
\end{equation}
\begin{equation}
dx = p_{0} - P * q_{0}
\end{equation}
\begin{table}[h]
\centering
\begin{tabular}{|l|l|}
\hline
p  & Punkt in der X-Koordinate                             \\ \hline
q  & Punkt in der Y-Koordinate                             \\ \hline
dx & Vector für die Verschiebung (REG\_BGxX und REG\_BGxY) \\ \hline
A  & Transformation von X zu Y Koordinate                  \\ \hline
P  & Transformation von Y zu X Koordinate                  \\ \hline
\end{tabular}
\caption{Erlaeuterung der mathematischen Varbiablen}
\label{mathavar}
\end{table}

Wenn der Verschiebungsvektor dx verwendet wird, dann wird eine Transformation um den Punkt $p_{0}$gemacht. Das führt zu der y-Punktkoordinate $q_{0}$im Bild. Die Multiplikation $p_{0}* P$ ist die Korrektur der X-Koordinate. Dies ist notwendig, um die Rotation um den Punkt $q_{0}$, anstatt um (0,0) zu ermöglichen. Im Beispiel sbb\_aff.c wird diese Transformation von der Funktion \textbf{bg\_rotscale\_ex()} übernommen.
\begin{figure}[h]
	\centering
	\includegraphics[height=85mm]{img/rotation.png}
	\caption{Rotationbeispiel der Tiles}
\end{figure}

Hier ist das Code-Snippet mit eingefügten Kommentaren, um die Verständnis des Codes zu erleichtern. Die Kommentare sind ergänzt und der Quellcode ist ausführlich mit dem mGBA getestet worden. Auch hier würde eine detailiertere Beschreibung den Rahmen dieser Studienarbeit sprengen.
\lstinputlisting[language=C, breaklines = true, numbers=left, frame=single, basicstyle=\ttfamily, keywordstyle=\color{blue}\ttfamily, stringstyle=\color{red}\ttfamily, commentstyle=\color{green}\ttfamily, label=sbbaff,caption=sbb\_aff() Funktion]{code/rotate.c}

\chapter{Rendering Mode 2}
Dadurch, dass das Team nur aus zwei Personen besteht, wird nur auf die Rendering Modes 0 und 1 in dieser Studienarbeit näher eingegangen und vom Mode 2 die Grundlagen für das allgemeine Verständnis erklärt. \\
Im Mode 2 gibt des zwei Schichten mit dem Namen Background 2 und 3 welche beide skalierbar und rotierbar sind.
Das Prinzip im Mode 2 ist das Gleiche, wie im Mode 1 der BG2, nur dass im Mode 2 noch mehr Flexibilität durch zwei skalierbare Hintergründe gegeben ist. \\
\begin{table}[h]
\centering
\begin{tabular}{|l|l|l|l|l|}
\hline
\textbf{Mode}           & \textbf{BG 0}          & \textbf{BG1}           & \textbf{BG2}                & \textbf{BG3}                \\ \hline
\multicolumn{1}{|c|}{2} & \multicolumn{1}{c|}{-} & \multicolumn{1}{c|}{-} & \multicolumn{1}{c|}{affine} & \multicolumn{1}{c|}{affine} \\ \hline
\end{tabular}
\caption{Rendering Mode 2}
\label{Rendering Mode 2}
\end{table}




%\input{Kapitel/5_TestAblauf}
%\input{Kapitel/7_Zusammenfassung}

% ===========================================================
% Literaturverzeichnis
% ===========================================================

\printbibliography
\pagenumbering{Roman} 
\setcounter{page}{5}

\end{document}
